\chapter*{LỜI MỞ ĐẦU}
\addcontentsline{toc}{chapter}{MỞ ĐẦU}

\label{Chapter0} 

Machine Learning đã và đang đóng góp một vai trò quan trọng trong tất cả các lĩnh vực trên toàn thế giới, mỗi năm có hàng ngàn báo cáo khoa học về đề tài này. Các công ty công nghệ hàng đầu như Google, Facebook, Amazon và Microsoft đang đầu tư mạnh mẽ vào học máy để phát triển các sản phẩm và dịch vụ mới. Ngoài ra, học máy cũng đang được sử dụng rộng rãi trong các lĩnh vực như y tế, tài chính, sản xuất và nhiều lĩnh vực khác để cải thiện hiệu suất và tối ưu hóa quy trình.

Những năm gần đây, khi mà khả năng tính toán của các máy tính được nâng lên một tầm cao mới và lượng dữ liệu khổng lồ được thu thập, Machine Learning đã tiến thêm một bước dài và Deep Learning ra đời. Deep Learning đã giúp máy tính thực thi những công việc phức tạp hơn như: phân loại cả ngàn vật thể trong các bức ảnh, tạo chú thích cho ảnh, bắt chước giọng nói và chữ viết của con người, giao tiếp với con người...

Theo số liệu thống kê của Google cho đến thời điểm hiện tại, về tình hình dịch bệnh trên toàn thế giới. 

Trên toàn thế giới có hơn 649.799.405 triệu ca nhiễm, đã bình phục 626.994.443 và tổng số ca tử vong đã lên đến 6.646.043 triệu ca. Con số ca nhiễm và số tử vong vẫn không có dấu hiệu ngừng tăng.

Riêng đối với Việt Nam tính ngày 01/06/2023 đã có tổng số ca nhiễm là 11.602.738 triệu ca, trong đó đã có 10.635.065 triệu ca đã bình phục và 43.203 ca đã tử vong. Mỗi ngày vẫn còn khoảng 240 ca nhiễm covid .  Đó là một con số không hề nhỏ và đáng bảo động.

Ở nước ta, từ đầu năm 2020 đến nay đã sảy ra nhiều ca tử vong đáng tiếc và là lời cảnh tỉnh trong công tác phòng chống dịch bệnh. Công dân Việt Nam còn khá chủ quan và lơ là về việc tự trang bị và bảo vệ sức khỏe bản thân.

Giống như dịch SARS và MERS trước đây, virus corona từ Vũ Hán cũng gây dịch bệnh viêm phổi cấp và thậm chí gây chết người. Triệu chứng coronavirus chủng mới thường không thể hiện rõ ở một số người trong thời gian ủ bệnh, một số người khác thì có biểu hiện phổ biến như sốt, ho khan, khó thở, đau cơ và mệt mỏi.

Các virus corona ở người bị nhiễm bệnh thường lây sang người khác qua những con đường dưới đây:

\begin{itemize}
	\item Lây qua đường máu.
	
	\item Không khí do ho và hắt hơi.
	
	\item Tiếp xúc với người bệnh như chạm hoặc bắt tay
	
	\item Chạm vào một vật hoặc bề mặt có virus rồi chạm tay vào mặt
\end{itemize}

Chúng ta nên trang bị đầy đủ khẩu trang, nước xịt khuẩn để đảm bảo an toàn cho bản thân và mọi người xung quanh.

Nghiên cứu cho thấy, việc sử dụng khẩu trang có thể hạn chế được sự lây lan và truyền bệnh qua đường hô hấp mỗi khi chúng ta giao tiếp hay hắc hơi ở khoảng cách.

COVID-19 được biết đến là một đại dịch toàn cầu với tốc độ lây lan nhanh, mức độ nguy hiểm cao với nhiều ca bệnh phức tạp. Các phương pháp phòng tránh dịch bệnh được quan tâm hàng đầu, đó là việc đeo khẩu trang. Tuy nhiên, không phải ai cũng hiểu được giá trị và tầm quan trọng của nó mang lại. Để giải quyết vấn đề đặt ra, nhóm 2 đã nghiên cứu, xây dựng đề tài "Ứng dụng mạng nơ-ron tích chập (Convolutional Neral Network) trong nhận diện khẩu trang và cảm xúc" với mong muốn mang lại giá trị cho cộng đồng, đánh giá cảm xúc mọi người và chung tay đẩy lùi dịch bệnh . 

Chính vì tất cả những lí do trên mà ứng dụng nhận diện đeo khẩu trang này được đưa ra để nghiên cứu và thực hiện.

Nội dung chính của báo cáo có 4 chương:

Chương 1: Tổng quan về đề tài

Chương 2: Cơ sở lý thuyết dùng để giải quyết bài toán

Chương 3: Chương trình thực nghiệm  

Chương 4: Kết quả và phương hướng phát triển  
 


