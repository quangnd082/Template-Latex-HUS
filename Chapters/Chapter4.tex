% Chương 4 

\chapter{KẾT QUẢ VÀ PHƯƠNG HƯỚNG PHÁT TRIỂN} 

\label{Chapter4}

\section{Kết quả}

Kết quả đạt được : Nhóm 2 đã ứng dụng thành công mạng neural tích chập trong việc phát hiện người có hay không đeo khẩu trang và nhận diện cảm xúc con người. Tuy nhiên dự án của nhóm vẫn còn một số những hạn chế nhất định, những ưu điểm và nhược điểm sau:

\subsection{Ưu điểm}

\begin{itemize}
	\item Tính ứng dụng: Tính ứng dụng cao, có thể linh hoạt sử dụng trong nhiều hoàn cảnh, địa điểm khác nhau. 
	
	\item Hiệu suất cao: CNN rất hiệu quả trong việc phân loại xác định hình ảnh, mô hình khẩu trang đã đạt được độ chính xác cao ( >98\%) trong việc phân loại người có hay không đeo khẩu trang.
	
	\item Xử lý được dữ liệu thời gian thực. 
\end{itemize}

\subsection{Nhược điểm}

\begin{itemize}
	\item Yêu cầu dữ liệu đủ lớn: Để đạt được hiệu suất tốt, mô hình CNN yêu cầu một lượng lớn dữ liệu. Điều này gây khó khăn trong việc phân loại cảm xúc khi mà lượng dữ liệu không được cân đối trong các lớp. 
	
	\item Tính chất dữ liệu có nhiều điểm chung: Khi thực hiện xác nhận có đeo khẩu trang, không đeo và đeo sai cách thì đeo sai cách rất khó phân biệt do có nhiều điểm giống với có đeo.
	
	\item Chưa được hoàn toàn chính xác và độ ổn định chưa được cao. 
\end{itemize}

\section{Hướng phát triển trong tương lai}

\begin{itemize}
	\item Trong tương lai, nhóm muốn tạo ra một app trên điện thoại để có thể đến được gần hơn với nhiều người. 
	
	\item Muốn phát triển lên thành sản phẩm hoàn chỉnh có tính thực tế cao hơn.	
	
	\item Không chỉ dừng lại ở việc xác định khẩu trang thêm vào đó là cảnh báo bằng âm thanh nếu thấy đối tượng không đeo khẩu trang. 
\end{itemize}



