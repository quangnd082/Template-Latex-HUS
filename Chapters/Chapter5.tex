\chapter*{KẾT LUẬN}
\addcontentsline{toc}{chapter}{KẾT LUẬN}

\label{Chapter5}

Bài Báo cáo đã khái quát tầm quan trọng của chiếc khẩu trang đối với sức khỏe của con người trong thời kì dịch bệnh có nhiều diễn biến phức tạp, khó lường. Bên cạnh đó cũng đã ứng dụng Machine Learning (Học máy) vào việc nhận diện người dân có hay không đeo khẩu trang hoặc đeo sai cách và nhận diện được cảm xúc của họ theo thời gian thực.

Bài Báo cáo cung cấp khái niệm của mạng neural tích chập (Convolutional Neural Network - CNN), nắm được những thành phần cơ bản của một mô hình mạng neural, kiến trúc của mạng CNN, áp dụng được các thư viện (Numpy, OpenCV, Matplotlib, Tensorflow,...), để giải quyết vấn đề đã đặt ra từ trước. 

Đánh giá và khảo sát mô hình một cách nghiêm túc, thực tế, rút ra được những tiêu chí, yếu tố ảnh hưởng tới mô hình (Độ rõ ràng của tập dữ liệu, Chọn mô hình phù hợp, Quá trình huấn luyện,...), đánh giá mô hình (Độ chính xác, Hàm mất mất-Loss, Tính đồng nhất,...),...

Bài Báo cáo mới chỉ dừng lại ở mức độ nghiên cứu, bước đầu chạy thử nghiệm nhận biết khẩu trang và cảm xúc thành công, mô hình còn nhiều điểm chưa được ổn định, còn đơn giản, không quá phức tạp, đáp ứng một số tiêu chí nhất định để phù hợp với định hướng và mục tiêu mô học đề ra. Định hướng tương lai có thể thực hiện thành công và đưa vào phục vụ đời sống xã hội.









